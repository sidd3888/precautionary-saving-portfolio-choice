\section{Discussion}\label{discussion}

% \subsection{Reconciling model predictions with data}\label{reconciliation}

Section \ref{us_data} shows that while a positive covariance between permanent income growth shocks and equity return shocks ($\o_{\psi,\,\eta}$) does explain the preference for saving in the safe asset to some extent, the expected returns on equity ($\Rfree$) in the data are too high at relatively low volatility ($\s_{\eta}$). As such, even with a very high correlation between permanent income growth shocks and equity returns, we require unrealistically high levels of relative risk aversion to explain the portfolio allocation observed in the data. A particular problem area is that the equity share limit for wealthy individuals is extremely high, at close to 80 percent, even with $\rho = 7$. While portfolio shares dip to reasonable numbers with zero income events around the target wealth, the model cannot explain the portfolio choice decisions of consumers outside a small neighborhood of the target wealth in the distribution and non-participation in the stock market.

One approach used to address this problem is to incorporate non-expected utility preferences. \citet{Haliassos2001} examine how various models of decision-making under risk improve predictions on equity holdings. While they conclude that changing preferences alone is not sufficient to account for the equity premium, they show that \citet{Kreps1978} preferences and, to a greater extent, Rank-Dependent Utility \citep{Quiggin1982} provide more realistic predictions on portfolio composition. Similarly, \citet{Schreindorfer2020} shows that disappointment averse preferences \citep{Gul1991, Routledge2010} can help explain the equity premium at a much lower level of relative risk aversion than with expected utility preferences under their model. However, the risk aversion coefficient necessary with expected utility preferences in their model is 34, meaning that despite the marked improvement, the new risk aversion coefficient is as high as 10.

Another explanation for the lower portfolio share of equity in the data than predicted in the model is pessimism and heterogeneity in beliefs about stock returns. \citet{Haliassos1995} argue that in addition to correlations between labor income and asset returns and departures from expected utility, factors such as informational frictions provide a good explanation for the equity premium. While they note that a lack of knowledge about the stock market constrained participation, \citet{Dominitz2007} find that agents also hold exaggerated beliefs about the possibility of negative nominal stock returns. \citet{Mateo2024} incorporates estimated beliefs from survey data into a life-cycle model and finds that stock market participation and conditional equity portfolio share can be explained by heterogeneous beliefs with a high average belief about the volatility of returns to equity. For consumers who believe that stock returns are extremely volatile, a high, or even moderate correlation between permanent income shocks and asset returns should ensure that they do not participate in the stock market, whereas the conditional distribution over equity portfolio share would then be determined by those who believe the stock market is not as volatile, though possibly more than actually observed in the data.

As far as non-participation in the stock market is concerned, minimum investment limits and fixed costs for participation have also been studied as probable obstacles. In the current model, with the parameterization as in section \ref{us_data} stock market non-participation is observed at target wealth solely due to the no-borrowing constraint, and cannot be seen with the zero income event. However, a fixed participation cost would preclude consumers with very little wealth from investing their savings in equity, thereby generating a non-participation effect among low-wealth consumers. One limitation to this approach is that it cannot explain non-participation across wealth levels. \citet{Andersen2011,Briggs2021} show that consumers who experience windfall gains do not see significantly higher participation rates, and that some of them even liquidate inheritances received in the form of stock.\footnote{Note that individuals who experience windfall gains do not experience an increase in permanent labor income, which means that their normalized wealth must also increase. Thus, variance in wealth due to the variance in permanent income alone does not capture such individuals.}