\section{Introduction}

The standard buffer stock model \citep{Deaton1991, Carroll1992} predicts that consumers engage in precautionary saving either because of borrowing constraints or the possibility of unemployment. Despite this, under a canonical formulation with independent shocks to income and returns on equity, the model counterintuitively predicts that poor consumers should invest their entire savings in equity, while wealthy consumers should save in the safe asset.\footnote{The one exception is that with zero income events, arbitrarily poor consumers invest the same fraction of their savings in equity as the arbitrarily wealthy.} The predictions of lifetime portfolio optimization models by \citet{Merton1969, Samuelson1969} suggest that the major component in the willingness to invest in the risky asset arises from a regular income stream \citep{Heaton1997}. 

\citet{Kimball1991} showed that consumers may display a temperance motive to moderate exposure to other risks when faced with income uncertainty, even if these risks are statistically independent. \citet{Koo1999} shows that an increase in the volatility of permanent income growth shocks significantly amplifies the temperance effect, though transitory income shocks have negligible influence. This does not change a key prediction of the model, however, that the poor should invest all their savings in equity, and the portfolio share of equity declines with wealth. Furthermore, given the equity premium and the volatility of returns on stock observed in the data \citep{Mehra1985}, the baseline model suggests that even the wealthy should invest all their savings in stocks.

One explanation for the modest portfolio share of equity observed in the data is that future consumption is positively correlated with returns on equity. \citet{Constantinides2002} use an overlapping generations (OLG) model to explore this idea. In their model, there are three generations of individuals: the young, middle-aged, and old/retired. Retired individuals receive a labor income of zero, which means that the future consumption of the middle-aged is highly correlated with the returns on equity. On the other hand, the middle-aged face uncertain wage income, implying that the future consumption of the young depends on more than just the returns on equity, lowering their correlation. Their argument is that a positive correlation implies that the realization of low marginal utility of consumption coincides with high returns, and vice versa. Thus, the low portfolio share of equity is a consequence of the inability of the young to participate in the stock market due to borrowing constraints. This paper departs from their modeling assumptions by retaining income uncertainty and introducing correlations between permanent income growth shocks and equity returns to correlate future consumption and equity returns. A positive correlation between these shocks reduces the potential of equity risk to serve as a hedge against fluctuations in future consumption due to permanent income growth risk.

This paper primarily focuses on an infinitely-lived agent\footnote{The setup is well-suited to life-cycle modeling, but the insights derived in this paper are from the predictions of the infinite-horizon model.} who consumes a single good and maximizes the discounted sum of utility from consumption. The agent faces three risks: shocks to the return on equity, permanent income growth, and transitory income. The model can accommodate pairwise correlations between all three shocks, though the focus of this paper will be on the case where permanent income shocks and equity returns are correlated, while transitory income shocks are independent of both. The agent decides their consumption (therefore, saving) and the portfolio share of equity in every period. The flexibility of the model in terms of the distribution of the shocks allows us to establish that a correlation between transitory income shocks and equity returns has a negligible effect on the portfolio share of equity for all agents but those with near-zero savings.

There is ample justification in the literature to model permanent income growth as correlated with equity returns while modeling transitory income shocks as independent of the two. \citet{Campbell1996} shows that there is a high correlation between the present value of human capital and market returns, despite finding that the contemporaneous correlation between wage income and stock returns is low. However, the source of this covariance in his model is due to a common time-varying discount factor applied to both calculations. \citet{Baxter1997} also find that the correlation between the returns to human capital and physical capital are high, even as labor and capital income growth rates may not exhibit a high correlation. With independent shocks across time, permanent income growth proportionally affects the expected present discounted value of human capital, while the independent transitory shock allows for the low correlation in actual labor income growth and returns on equity.

There have been papers that have studied correlations between labor income growth and equity returns previously in a life-cycle setting. \citet{Bodie1992} modelled the labor supply decision as endogenous, and made after the current return to equity has been determined. They show that agents vary their labor supply ex-post to cushion themselves against greater risks taken in their investment decisions. Subsequent papers study this relationship with an exogenous income process and a positive relationship between income growth and equity returns. \citet{Benzoni2007} explore a model with cointegrated labor income and stock market returns in a continuous-time setting. \citet{Bagliano2014} on the other hand study a discrete-time model with a permanent income shock that has both an aggregate and an idiosyncratic component, and two risky assets. The closest paper to mine is by \citet{Viceira2001}, whose model is based on a similar buffer-stock setting with correlations between permanent income growth and risky asset returns. The difference here is that I focus on the problem of the infinitely-lived agent, whose income is subject to both transitory and permanent income shocks, and who may face an artificial borrowing constraint instead of possible spells of unemployment that otherwise serves as the motive for precautionary saving. My focus, thus, is on portfolio choice across the wealth distribution and around the target wealth, as opposed to over the life-cycle.\footnote{\citet{Fagereng2020} point out that the composition of the portfolio even within the class of risky assets is heterogeneous across the wealth distribution. This affects the rate of return on the risky component of the portfolio, which, in turn, also affects portfolio choice.}

The saving behavior of the agent who faces an artificial borrowing constraint is slightly different as compared to that of the agent who faces a positive probability of zero income. While both factors serve as precautionary saving motives, the consumer who faces artificial borrowing constraints tends to save very little at their target wealth and therefore invests none of their savings in equity, which lends support to the well-studied argument that borrowing constraints among the young, for instance, can cause reduced or non-participation in the stock market \citep{Constantinides2002, Haliassos2002, Kogan2007, Jang2015, Harenberg2018}. I also artificially restrict the portfolio share of equity to between 0 and 1, which implies that consumers cannot supplement their portfolio with debt-financed equity positions. \citet{Davis2006} argue that borrowing costs can be a major deterrent against investing in equity using loans, and it is beyond the scope of this paper to examine the differences in rates of return and mechanisms available for consumers to borrow versus save.

The findings of this paper can be divided into three major strands. First, a positive correlation between permanent income growth shocks and returns on equity does have a negative effect on the portfolio share of equity of an agent with a moderate level of risk aversion ($\rho = 4$). When faced with the no-borrowing constraint, this effect is most prominently observed among the poor, who save very little and invest all of their savings in the risk-free asset. Proposition \ref{prop:low_wealth_share} also shows that this effect is indeed discontinuous in the covariance between permanent income growth and equity return shocks. In fact, poor agents invest either all or none of their savings in equity, and they invest none if the covariance between income growth and equity return exceeds the equity premium times the inverse of the expected risky rate, scaled by the RRA parameter. This effect is absent in the model where a positive zero-income probability serves as the motive for precautionary saving instead of the borrowing constraint. This effect gradually disappears in wealth, as the limiting portfolio share of equity approximately tends to the optimal portfolio share in the models by \citet{Merton1969,Samuelson1969}, which is independent of the correlation between income and equity return shocks. Proposition \ref{prop:high_wealth_share} formalizes this result for both the cases where the agent faces a no-borrowing constraint and a positive probability of zero-income.

Second, in a closely related effect, the optimal portfolio share of equity is increasing in wealth for above-threshold levels of covariance between the permanent income growth and equity return shocks. In particular, Proposition \ref{prop:wealth_share_diff} shows that in the model with the no-borrowing constraint, this effect is perfectly tied with the threshold level of covariance that makes the poor invest in the risk-free asset. Numerical approximations and further analysis reveal why this conclusion eludes the model with zero-income events. In particular, as the agent becomes arbitrarily poor, the incidence of the zero-income event on utility differences becomes infinitely pronounced. Since the zero-income event occurs independently of all other shocks, the agent accords highest weight to sequences of zero-income shocks in the limit, causing portfolio choice behavior to revert to that in the Merton-Samuelson model.

Third, with the artificial borrowing constraint, agents do not invest in equity or invest extremely small proportions of their savings in it around the target level of wealth, while the threat of unemployment better explains the portfolio share of equity around target levels of wealth conditional on participation in the stock market. Though the model generates insightful predictions for an equity premium set at 3 percent with the standard deviation of the logged shock to the return set at 15 percent, calibrating the model to parameters implied by data on U.S. equity returns as reported in \citet{Mehra2006} necessitates an elevated risk aversion parameter of around 7 to generate modest portfolio shares of equity around target wealth.

The covariance of permanent income growth and returns on equity is, of course, one among multiple explanations posited to resolve the equity premium puzzle. A section of the literature argues that habit-formation makes consumers more averse to the volatility in consumption across time-periods induced by the risk on equity return \citep{Constantinides1990, Abel1990, Detemple1991, Campbell1999, Otrok2002}. Another strand of the literature focuses on the low participation rates in the stock market despite the large equity premium and uses market frictions like fixed participation costs to explain increasing participation with age and wealth \citep{Cocco2005, Gomes2005, Alan2006, Khorunzhina2013, Fagereng2017}. Modest portfolio shares of equity are also explained by heterogeneous and pessimistic beliefs about the returns on stocks. The evidence suggests that beliefs about the volatility of returns on equity are exaggerated, and portfolio decisions are correlated with the reported beliefs about the stock market \citep{Dominitz2007, Hurd2011, Amromin2014, Ameriks2020, Mateo2024}. Yet another explanation revolves around heterogeneity in preferences or departures from expected utility \citep{Guvenen2009,Haliassos1995,Haliassos2001, Routledge2010, Schreindorfer2020}.\footnote{See \citet{Barucci2003} for a detailed exploration of the assumptions imposed under the standard rational paradigm and the puzzles that stem from relaxing them.} Section \ref{discussion} discusses how some of these explanations may tie in with the model in this paper to favorably revise the predictions of the model without requiring a large coefficient of relative risk aversion.

Finally, numerically solving a model with three potentially correlated shocks presents its own computational challenges. Numerical integration, particularly given a joint distribution of more than two random variables, can be painfully slow. To that end, one of the contributions of this paper is to provide a simple algorithm to discretize a multivariate lognormal distribution to compute expectations in the Euler equation. This algorithm is detailed in Section \ref{discretization}.

The paper is structured as follows. Section \ref{model} sets up the model and provides a preliminary discussion of the solution to the model. Section \ref{results} provides the main results and analysis contained in the paper. Section \ref{us_data} examines the implications of the calibrated temperance motive for portfolio choice in the model. Section \ref{discussion} provides a few comments and concludes.