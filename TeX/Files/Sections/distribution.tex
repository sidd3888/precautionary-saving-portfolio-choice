\section{Distribution of wealth and portfolio share}

The results in the previous section highlight the nuances of the how consumers in the model optimally allocate their savings between the risk-free asset and equity. However, another informative facet of the model is the stationary distribution over wealth and portfolio share that the behavior of agents gives rise to. The distribution of wealth and portfolio share is important in that it provides a richer picture of how the precautionary saving motives of agents interact with the portfolio choice problem.

Given the form of the policy function, note that under either of the specifications, computing a wealth distribution of agents is tantamount to obtaining a distribution over the portfolio share of equity. To do so, we can simulate the trajectories of $N$ agents (assuming a suitably large $N$) over a substantial number of periods to approximate a stationary distribution of wealth and portfolio shares. The first thing we need to do prior to this exercise is to formalize the manner in which shocks are generated.

Firstly, returns on equity are common to all agents. The shocks to permanent income growth and transitory income, however, can be modelled as idiosyncratic. Then, a consumer's wealth transition is given by
\[
m_{i,\,t+1} = \frac{\Rfix + \vs(m_{i,\,t})(\Rfree\eta_{t+1} - \Rfix)}{\G\psi_{i,\,t+1}}(m_{i,\,t} - c(m_{i,\,t})) + \t_{i,\,t+1}
\]
Note that the shocks to income experienced by all the individuals are independent of each other only when the shocks to income are uncorrelated to the common shock, which is the return on equity. However, the shocks to permanent and transitory income can be modelled as conditionally independent upon the realization of the shock to the return on equity. Then, the transition of wealth in the economy can be simulated by generating an asset return shock for each period, and generating $N$ values for the permanent income growth shock from the conditional distribution of $\psi$ and $\z$, given the realization of $\eta$. In the NBC model, the generated values of $\z$ are exactly the values of $\t$. In the NIR model, however, we can independently generate $N$ draws of a Bernoulli random variable with probability $1 - \wp$ to determine which agents experience a zero-income event, thereby determining the value of $\t$. Given the realization of the shocks, the transition will be specified by $m_{i,\,t+1}$ as defined above.

For the general statement of the problem, my simulation algorithm proceeds as follows. I first generate a single draw from the marginal distribution of the return on equity. Then, given the realized value of $\ln(\eta)$, I calculate the conditional distribution of the bivariate normal variables $(\ln(\psi),\,\ln(\z))$ as in \citet[p.34]{Anderson2003}, and instantiate the derived bivariate lognormal. I then use the draws from this distribution and the Bernoulli draws to determine $\psi_{i,\,t+1}$ and $\t_{i,\,t+1}$, for each $i$. Finally, I calculate the transition of wealth for each agent, and repeat the process for a large number of periods ($T = 120$) to approximate the stationary distribution of wealth and \textit{incoming} portfolio shares. That is, the distribution over portfolio shares of equity stemming from previous period wealth, which reflects each agent's beginning-of-period asset holdings. I look at the baseline parameterization used in Section \ref{results}.

\begin{figure}[h]
    \centering
    \begin{subfigure}{0.49\textwidth}
        \centering
        \includegraphics[width=0.8\textwidth]{\NBCwealthdist}
        \caption{NBC model}
        \label{subfig:NBCwealthdist}
    \end{subfigure}
    \begin{subfigure}{0.49\textwidth}
        \centering
        \includegraphics[width=0.8\textwidth]{\NIRwealthdist}
        \caption{NIR model}
        \label{subfig:NIRwealthdist}
    \end{subfigure}
    \caption{Stationary wealth distribution in the NBC and NIR models}
    \label{fig:wealthdist}
\end{figure}
Figure \ref{fig:wealthdist} shows the distribution of wealth-to-permanent income among individuals in the NBC and NIR models. As is already well understood, the stationary distribution of wealth is centered around the target level of wealth in both models, though these savings targets are different, due to the differences in how agents respond to borrowing constraints and zero-income events. Given this stationary distribution, we can define a stationary distribution over portfolio shares of equity, which is a function of the wealth-to-permanent income ratio for each individual. The stationary distribution of portfolio shares in shown in Figure \ref{fig:sharedist}. Both distributions are radically different. Due to the nature of the optimal portfolio share rule in the NBC model, there is a high rate of clustering around 0. That is, many agents do not participate in the stock market, so long as they do not face a large transitory income shock in the current period. In the NIR model, however, the distribution is more spread out, reflecting the smoothness of the portfolio share rule around the target level of wealth.
\begin{figure}[h]
    \centering
    \begin{subfigure}{0.49\textwidth}
        \centering
        \includegraphics[width=0.8\textwidth]{\NBCsharedist}
        \caption{NBC model}
        \label{subfig:NBCsharedist}
    \end{subfigure}
    \begin{subfigure}{0.49\textwidth}
        \centering
        \includegraphics[width=0.8\textwidth]{\NIRsharedist}
        \caption{NIR model}
        \label{subfig:NIRsharedist}
    \end{subfigure}
    \caption{Stationary distribution of portfolio shares in the NBC and NIR models}
    \label{fig:sharedist}
\end{figure}
The joint distribution of wealth and equity portfolio share going into the next period is rather trivial, as it is concentrated along the optimal portfolio share rule, weighted according to the stationary distribution of wealth. On the other hand, there is merit in examining the joint distribution of current portfolio holdings (which are determined by the previous period's wealth), and current wealth. This distribution is shown in Figure \ref{fig:jointdist}. As predictable, this distribution exhibits the relationship between wealth and portfolio share according to the optimal portfolio share rule around target wealth, but also reflects the dispersion arising from idiosyncratic shocks in the next period. Furthermore, the distribution in the NIR model also reflects the small proportion of agents who are faced with zero-income events, who, despite their lower wealth, invest similar amounts in equity as those with higher wealth.
\begin{figure}[h]
    \centering
    \begin{subfigure}{0.49\textwidth}
        \centering
        \includegraphics[width=0.8\textwidth]{\NBCjointdist}
        \caption{NBC model}
        \label{subfig:NBCjointdist}
    \end{subfigure}
    \begin{subfigure}{0.49\textwidth}
        \centering
        \includegraphics[width=0.8\textwidth]{\NIRjointdist}
        \caption{NIR model}
        \label{subfig:NIRjointdist}
    \end{subfigure}
    \caption{Stationary joint distribution of wealth and portfolio shares in the NBC and NIR models}
    \label{fig:jointdist}
\end{figure}