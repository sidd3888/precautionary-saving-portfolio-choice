\begin{titlepage}
    \maketitle\thispagestyle{empty}
    \vspace*{\stretch{0.5}}
    \begin{abstract}
        Adding a portfolio choice problem to the consumption-saving model of \citet{Deaton1991} or \citet{Carroll1992} shows that the optimal portfolio share of equity declines with wealth, even in the presence of precautionary saving motives. Glaringly, the poor invest all their savings in equity. Using an infinite-horizon consumption-saving problem, I examine the effects of a positive covariance between permanent income growth and returns on equity \citep{Viceira2001} on optimal portfolio allocation. A positive covariance between the two decreases the optimal portfolio share of equity at all wealth levels (\citepos{Kimball1991} temperance motive). If the precautionary saving motive is a no-borrowing constraint, with a positive covariance between permanent income growth and returns on equity beyond a threshold value that depends only on the equity premium percentage and risk aversion (i) the optimal portfolio share of equity increases with wealth, (ii) the poor invest all their wealth in the risk-free asset, and (iii) the optimal portfolio share is lower than the Merton-Samuelson (no labor income) share. If the precautionary saving motive is a positive probability of zero labor income, however, (i) and (ii) do not hold true, though (iii) does. When calibrated to U.S. data, neither precautionary saving motive can fully explain observed portfolio shares of equity even with a very high covariance between permanent income growth and returns on equity. The calibrated temperance motive may, however, explain the observed equity portfolio share in conjunction with other popular explanations documented in the literature.
    \end{abstract}
    \vspace*{\stretch{0.5}}
    Link to code: \href{\codeurl}{\texttt{\codeurl}}
    \vspace*{\stretch{1.5}}
\end{titlepage}