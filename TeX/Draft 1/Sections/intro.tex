\section{Introduction}

The standard buffer stock model \citep{Deaton1991, Carroll1997} predicts that consumers engage in precautionary saving either because of borrowing constraints or the possibility of unemployment. Despite this, under a canonical formulation with independent shocks to income and returns on equity, the model counterintuitively predicts that poor consumers should invest their entire savings in equity, while wealthy consumers should save in the safe asset.\footnote{The one exception is that with zero income events, extremely poor consumers invest a lower fraction of their savings in equity than consumers at the target level of wealth.} The predictions of lifetime portfolio optimization models by \citet{Merton1969, Samuelson1969} suggest that the major component in the willingness to invest in the risky asset arises from a regular income stream. \citet{Koo1999} shows that an increase in the volatility of permanent income growth shocks makes the consumer more cautious about investing in the risky asset, whereas transitory income shocks have a negligible effect on the same. However, this does not change the key prediction of the model that the poor should invest all their savings in equity, and the portfolio share of equity declines with wealth. Furthermore, given the equity premium and the volatility of returns on stock observed in the data, the baseline model suggests that even the wealthy should invest all their savings in stock.

One explanation for the modest portfolio share of equity observed in the data is that future consumption is positively correlated with returns on equity. \citet{Constantinides2002} use an overlapping generations (OLG) model to explore this idea. In their model, there are three generations of individuals: the young, middle-aged, and old/retired. Retired individuals receive a labor income of zero, which means that the future consumption of the middle-aged is highly correlated with the returns on equity. On the other hand, the middle-aged face uncertain wage income, implying that the future consumption of the young depends on more than just the returns on equity, lowering their correlation. Their argument is that a positive correlation implies that the realization of low marginal utility of consumption coincides with high returns, and vice versa. Thus, the low portfolio share of equity is a consequence of the inability of the young to participate in the stock market due to borrowing constraints. This paper departs from their modeling assumptions by retaining income uncertainty and introducing correlations between permanent income growth shocks and equity returns to correlate future consumption and equity returns.

This paper primarily focuses on an infinitely-lived agent\footnote{The setup is well-suited to life-cylce modeling, but the insights derived in this paper are from the predictions of the infinite-horizon model.} who consumes a single good and maximizes the discounted sum of utility from consumption. The agent faces three risks: the return on equity, permanent income growth, and transitory income. The model can accommodate pairwise correlations between all three shocks, though the focus of this paper will be on the case where permanent income shocks and equity returns are correlated, while transitory income shocks are independent of both. The agent decides their consumption (therefore, saving) and the portfolio share of equity in every period. The flexibility of the model in terms of the distribution of the shocks allows us to establish that a correlation between transitory income shocks and equity returns has a negligible effect on the portfolio share of equity for all agents but those with near-zero savings.

There is ample justification in the literature to model permanent income growth as correlated with equity returns while modeling transitory income shocks as independent of the two. \citet{Campbell1996} shows that there is a high correlation between the present value of human capital and market returns, despite assuming that the contemporaneous correlation between wage income and stock returns. However, the source of this covariance in his model is due to a common time-variant discount factor applied to both calculations. \citet{Baxter1997} also find that the correlation between the returns to human capital and physical capital are high, even as labor and capital income growth rates may not exhibit a high correlation. With independent shocks across time, permanent income growth proportionally affects the expected present discounted value of human capital, while the independent transitory shock allows for the low correlation in actual labor income growth and returns on equity.

There have been papers that have studied correlations between permanent income growth and asset returns previously in a life-cycle setting. \citet{Benzoni2007} explore a model with cointegrated labor income and stock market returns in a continuous-time setting. \citet{Bagliano2014} on the other hand study a discrete-time model with a permanent income shock that has both an aggregate and an idiosyncratic component, and two risky assets. The closest paper to mine is by \citet{Viceira2001}, whose model is based on a similar buffer-stock setting with correlations between permanent income growth and risky asset returns. The difference here is that I focus on the problem of the infinitely-lived agent, whose income is subject to transitory and permanent income shocks, and who may face an artificial borrowing constraint instead of possible spells of unemployment that otherwise serves as the motive for precautionary saving.

The imposition of the artificial borrowing constraint serves multiple purposes. First, in a model trying to understand the long-term equity holdings of consumers, it abstracts away from short-selling behavior. Then, it avoids any discontinuities in the problem at the point where savings are zero. Most lucratively, however, the consumer who faces artificial borrowing constraints tends to save very little at their target wealth and therefore invests none of their savings in equity, lending support to the well-studied argument that borrowing constraints among the young, for instance, can cause reduced or non-participation in the stock market \citep{Constantinides2002, Haliassos1998, Kogan2007, Jang2015, Harenberg2018}. I also artificially restrict the portfolio share of equity to between 0 and 1, which implies that consumers cannot hold debt-financed equity positions. \citet{Davis2006} argue that borrowing costs can be a major deterrent against investing in equity using loans, and it is beyond the scope of this paper to examine the differences in rates of return and mechanisms available for consumers to borrow versus save.

The findings of this paper can be divided into three major strands. First, a positive correlation between permanent income growth shocks and returns on equity does have a negative effect on the portfolio share of equity of an agent with a moderate level of risk aversion. This effect is primarily observed among the poor who save very little, and gradually disappears for extremely wealthy agents. As wealth tends to infinity, the portfolio share of equity tends to the value in the case with uncorrelated income and return shocks. Second, for high levels of correlation between the shocks, the portfolio share of equity is actually increasing in wealth, unlike the model with uncorrelated shocks. Third, with the artificial borrowing constraint, agents do not invest in equity or invest extremely small proportions of their savings in it around the target level of wealth, while the threat of unemployment better explains the portfolio share of equity around target levels of wealth conditional on participation in the stock market. Though the model generates insightful predictions for an equity premium set at 3 percent with the standard deviation of the logged shock to the return set at 15 percent, calibrating the model to parameters relevant to data on U.S. equity returns as reported in \citet{Mehra2006} necessitates an elevated risk aversion parameter of around 7 to generate modest portfolio shares of equity around target wealth. Section \ref{reconciliation} discusses other explanations for the equity portfolio share that, in conjunction with correlated shocks to permanent income and returns, can favorably revise the predictions of the model without requiring a large coefficient of relative risk aversion.

The paper is structured as follows. Section \ref{model} sets up the model and discusses the solution method used. Section \ref{results} examines predictions from the model with the artificial no-borrowing constraint. Section \ref{zero_inc} relaxes the borrowing constraint and introduces the possibility of unemployment in the income process. Section \ref{discussion} provides a few comments and concludes.